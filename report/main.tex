\documentclass{article}
\usepackage{graphicx}
\usepackage[a5paper, margin=15mm]{geometry}

\title{Lab 7 - Project Report}
\author{AI24BTECH11031 - Shivram S\\
AI23BTECH11022 - Dhadheechi Ravva}
\date{}

\begin{document}
\maketitle
\tableofcontents
\pagebreak

\section{Introduction}

As part of the CS2323 Computer Architecture course, we had to extend the RISC-V 
simulator developed in Lab 4 to support cache simulation. This report outlines
the design of the cache simulator.

\section{Usage}

The project includes a \texttt{Makefile} which can be used to build the project and
test it. To build, run:
\begin{verbatim}
$ make
\end{verbatim} 
This produces an executable called \texttt{riscv\_asm} in the project directory.

The simulator can be run using the following command:
\begin{verbatim}
$ ./riscv_asm
\end{verbatim}

\section{Design of The Cache}


The simulator's cache consists of several \textbf{lines}. Each line consists of several
\textbf{entries}, and each entry holds a block of memory. 

The cache has a \textbf{replacement policy} which determines the block to be replaced in
case of a conflict, and a \textbf{write-back policy}, which tells it how to handle writes.

The \texttt{Cache} struct holds all data related to the simulator's cache

\begin{verbatim} 
typedef struct Cache {
    // Cache configuration
    size_t num_lines, block_size, associativity;
    enum {WRITEBACK, WRITETHROUGH} write_policy;
    enum {FIFO, LRU, RANDOM} replacement_policy;
    
    size_t hits, misses, writebacks; // Statistics

    size_t monotime; // For maintaining insert and access times
    uint8_t *mem; // Simulator memory 
    CacheLine *lines; // The actual cache data
    FILE *output_file; // File for logging accesses
} Cache;
\end{verbatim}




\end{document}